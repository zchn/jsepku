\documentclass[14pt,a4paper]{article}

\usepackage{zhfontcfg}
\usepackage{listings}
\usepackage{xcolor}

\title{J Search Engine 原理}
%\subtitle{PROJ3 实习报告}
\author{00648333 陈志杰(Joyan) \and 00648332 揭忠(lezhengyi)}
%\institute{北京大学计算机系}

\begin{document}

\lstset{numbers=left,
  frame=shadowbox,
  escapeinside={`}{`},
  breaklines,
  backgroundcolor=\color[rgb]{0.90,0.90,0.90},
  basicstyle=\small,
  stringstyle=\ttfamily,      % typewriter type for strings
  showstringspaces=false}     % no special string spaces
\maketitle        
\section{分组情况}
陈志杰(00648333),揭忠(00648332)
\section{工作流程}
\subsection{抓取部分}
\begin{description}
\item[文件名] crawl
\item[用法] crawl dir-name
\item[说明] 递归访问dir-name下的文件,在dir-name下建立.tianwang.raw原始文件。
\item[备注] 天网文件格式:\\
  \begin{lstlisting}
    version:1.0
    url:foo.txt
    date:Tue, 15 Apr 2003 08:13:06 GMT
    length:12345

    XXXXXXXXXXX
    XXXXXXXXXXXXX
    ...
    XXXXXX

    version:1.0
    ...
  \end{lstlisting}
\end{description}
\subsection{索引部分}
\begin{description}
\item[文件名] docoff
\item[用法] docoff xxx.raw
\item[说明] 将raw中每个原始文件的url的MD5与其正文在raw文件中的偏移关联,存入xxx.didx
\item[备注] *.didx文件格式:\\
  \begin{lstlisting}
    DocID offset
    DocID offset
    ......
  \end{lstlisting}
\end{description}
\begin{description}
\item[文件名] rawseg
\item[用法] rawseg xxx.raw xxx.didx
\item[说明] 根据xxx.didx对 xxx.raw 进行切词处理,存入 xxx.raw.seg
\item[备注] *.raw.seg文件格式:\\
  \begin{lstlisting}
    DocID term1 term2 ...
    DocID term1 term2 ...
    ......
  \end{lstlisting}
\end{description}
\begin{description}
\item[文件名] sort
\item[用法] sort xxx.didx > xxx.didx.sort
\item[说明] 排序
\end{description}
\begin{description}
\item[文件名] preinvert
\item[用法] preinvert xxx.raw.seg
\item[说明] 将xxx.raw.seg转化为准倒排文件xxx.piidx
\item[备注] *.piidx文件格式:\\
  \begin{lstlisting}
    term(lock 8char) DocID(lock 8char HEX)
    term DocID
    ......
  \end{lstlisting}
\end{description}
\begin{description}
\item[文件名] sort
\item[用法] sort xxx.piidx >xxx.piidx.sort
\item[说明] 排序
\end{description}
\begin{description}
\item[文件名] invert
\item[用法] invert xxx.piidx.sort
\item[说明] 将准倒排文件转化为正式的倒排文件,存入xxx.iidx
\item[备注] *.iidx文件格式:\\
  \begin{lstlisting}
    term DocID DocID docid ...
    term DocID DocID ...
    ......
  \end{lstlisting}
\end{description}
\subsection{查询部分}
\begin{description}
\item[文件名] query
\item[用法] query keyword
\item[说明] 将keyword切词,分别求出其docid集合,合并后将docid按照词频排序,输出到res.txt文件中
\item[备注] res.txt文件格式:\\
  \begin{lstlisting}
    docid offset1 offset2 offset3 ...
    docid offset1 ...
    ......
  \end{lstlisting}
\end{description}
\subsection{用户部分}
\begin{description}
\item[文件名] query.htm
\item[用法] 用浏览器打开
\item[说明] 将res.txt翻译成html格式并显示出来。
\item[备注] 条目格式:\\
  \begin{verbatim}
    文件名    
    摘要摘要摘要摘要摘要摘要摘要摘要摘要
    摘要摘要摘要摘要摘要摘(关键词高亮)    
    文件路径
  \end{verbatim}
\end{description}
\end{document}
%%% Local Variables: 
%%% mode: latex
%%% TeX-master: t
%%% End: 
