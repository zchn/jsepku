\documentclass[12pt,a4paper]{article}

\usepackage{zhfontcfg}
\usepackage{listings}
\usepackage{xcolor}

\title{J Search Engine}
\author{00648333 陈志杰 \and 00648332 揭忠}

\def\keyword[#1] {{\bf {#1}}}
\def\remark[#1] {{\sf {#1}}}



\begin{document}

\lstset{numbers=left,
  frame=shadowbox,
  escapeinside={`}{`},
  breaklines,
  backgroundcolor=\color[rgb]{0.90,0.90,0.90},
  basicstyle=\small,
  stringstyle=\ttfamily,      % typewriter type for strings
  showstringspaces=false}     % no special string spaces
\maketitle
\section{简介}
J Search Engine是由陈志杰(Joyan),揭忠共同开发的一个模块化,低耦合度,高扩展性的微星实验性本地文本搜索引擎,目的在于展示搜索引擎的基本原理和结构,熟悉可扩展性软件设计方法.
\section{使用方法}
\begin{enumerate}
\item pre\_process的第一个参数是需要递归建立索引的子目录,第二个参数是工程名。
\item 建立倒排索引成功后,会自动调用服务程序,监听本地的7891端口等待连接。
\item 如果早就建立好了索引,也可以直接调用"start.sh 目录名 工程名"启动服务。
\item 启动后,用telnet(windows底下的不支持,用linux下的)连接主机的7891端口即可。
\end{enumerate}

\section{主要流程说明}

\subsection{抓取部分}
\keyword[抓取:]
某根目录下的所有的文件遍历并且结合成一个比较大的.raw类型的文件(详情请参考crawl.cpp),.raw类型的文件格式如下:
\begin{lstlisting}
  version:1.0
  url:dir/path/foo.txt
  length:12345
  
  XXXXXXXXXXX
  XXXXXXXXXXXXX
  ...
  XXXXXX
  
  version:1.0
  ...
\end{lstlisting}
\remark[建立此文件以后就不需要再对原文件进行重新访问了,这是为了模仿实际情况,在实际的网络查询中,也不会直接去访问原文件。]
\subsection{预处理部分}
\begin{enumerate}
\item \keyword[切词:]
  将每个文件都切成一个一个的词,以固定的格式储存在.raw.seg类型的文件中。具体格式如下:
  \begin{lstlisting}
    DocID term1 term2 ...
    DocID term1 term2 ...
  \end{lstlisting}
\item 利用docoff处理raw文件,记录文档号和对应的偏移到.didx文件中,didx格式如下:
  \begin{lstlisting}
    DocID 十六进制的偏移量(例如:00002432)
    DocID 十六进制的偏移量
  \end{lstlisting}
\item 利用didx文件改进raw.seg文件,改成准倒排文件的格式(也就是 .piidx格式),主要格式如下:
  \begin{lstlisting}
    term1 DocID
    term2 DocID
    ...
  \end{lstlisting}
\item 利用linux自带的sort工具对piidx文件按照单词排序,整理成 .piidx.sort文件。
\item 利用invert在piidx.sort的基础上再一步改进,改成倒排文件的格式(也就是 .iidx格式),主要格式如下:
  \begin{lstlisting}
    term DocID DocID Docid...
    term DocID DocID ...
  \end{lstlisting}
\end{enumerate}
\remark[以上便是预处理的主要过程了,此步骤虽然简单,但是其优点也就是简单,在下面的过程中你将会看到一个又一个非常非常简单的步骤,但当其结合起来时确实现了一个比较复杂的功能。]

\subsection{查询服务器}
为了只调用一次create\_map()函数,减少查询的时间,该部分被我们做成了一个服务器,监听在7891端口,如果向他发送一个不以@打头的字符串,他就会启动搜索并返回结果.这个过程本应是udp的,但那样还要开发client端,这对于一个搜索引擎来说是无意义的(因为我们打算将来的2.0版本是直接基于B/S的),搜以我们暂时采用了tcp传输.\\
借到一个查询请求后,serv工作流程如下::\\
首先将预处理的文件利用map\_creat函数将每个词与其在文档中的位置对应起来,这样便于以后的查询。\\
map类能够很好的将记录和记录的位置对应起来,对以后的查询有非常大的帮助。\\
然后,当输入某一短语的时候,首先调用cut\_word函数对其进行切词处理,将其切成一个一个的词组(所以用户查询的时候只需要像百度那样连续输入即可),每个词组之间以空格隔开。\\
接着,将此第一个词组的读入到string中,利用map查找到记录的地址,然后调用get\_term函数从相应的地址中读取所需的内容,接着调用process\_term函数计算此词组的权值,并储存在vector数组中,调用res\_merge函数对此所有词组的vector数组进行处理,计算出所输入短语在每篇文章中的权值,然后serv函数按照权值大小将查询结果按照固定的模式输出。\\
输出格式为:
\begin{verbatim}
  
  标题(即文件名)
  摘要
  路径
  
  标题(即文件名)
  摘要
  路径
  
  标题(即文件名)
  摘要
  路径

\end{verbatim}
其中,对摘要中前后切词中文的处理是一个技术难点.
\section{本作品的亮点}
\begin{enumerate}
\item \keyword[模块化和可扩展性:]
  在设计和实现的时候,我们充分考虑到以后对她的扩展,所以尽量降低各组成部分之间的耦合度,简化接口,是各部分都能独立作为一个小工具.
\item \keyword[服务器化:]
  虽然简陋,但是却节省了建立查询树的时间,同时起到了很好的试验目的,为下一个基于B/S的并行版本的推出打下了结构和设计上的基础.
\end{enumerate}
\section{下个版本的改进}
\begin{enumerate}
\item \keyword[对中文的统一转换:]
  目前的版本只能处理gb2312编码的文本和文件名,从下个版本开始,我们打算将所有的编码在抓取的时候就统一转化为UTF-8处理,这样就可以支持多种编码格式了。
\item \keyword[进一步服务器化:]
  将serv拆分为数据库服务器和用户交互服务器,前者加入并发特征,只负责对提交的查询请求返回根据权重排好序的文档id,后者接受用户连接,转发请求和生成摘要并呈现给用户等特征。
\item \keyword[改进排名算法:]
  将单词第一次出现在文章中的权重设为1000,以后每重复出现一次就加一,同时将只检索到部分词的文章也加入排名。
\end{enumerate}
\section{SVN地址}
{\bf http://code.google.com/p/jsepku/}
\section{Bug Report}
"Joyan"<ilcq@163.com>
\section{分工}
\begin{description}
  \item[陈志杰(Joyan)]
    shell编程部分,crawler部分,切词部分,cut\_word函数,process\_term函数,res\_merge函数,serv.cpp文件,软件调试,文档编写。
  \item[揭忠]
    建立didx文件,piidx文件和iidx文件,query函数,create\_map函数,get\_term函数,源代码注释,文档编写。
\end{description}
\section{声明}
切词部分使用的是闫宏飞老师的TSE部分的切词实现,并做了少量修改(已在源文件中注明)。
\section{个人心得}
揭忠:通过此次大作业的编写,我充分体会到了将一个复杂问题简单化的方法,我认为此次作业的步骤划分绝对是一个亮点,以前总是认为一个函数实现的功能越多越好,通过此次的编写,才感觉到其实是越简单越好,充分感觉到步骤的重复性以及中间文本的重用。此次大作业通过对一些步骤的重用,将某些本来很复杂的程序很简单的实现了。
\end{document}
%%% Local Variables: 
%%% mode: latex
%%% TeX-master: t
%%% End: 
